%%%%%%%%%%%%%%%%%%%%%%%%%%%%%%%%%%%%%%%%%%%%%%%%%%%%%%%%%%%%%%%%%%%
\muemex is \muelu's interface to the MATLAB environment. It allows access
to a limited set of routines either \muelu as a preconditioner,
Belos as a solver and Epetra or Tpetra for data structures.
It is designed to provide access to \muelu's aggregation and
solver routines from MATLAB and does little else. \muemex allows users to
setup and solve arbitrarily many problems, so long as memory suffices.
More than one problem can be set up simultaneously.

\section{Cmake Configure and Make}\label{sec:muemex:cmake}
To use \muemex, Trilinos must be configured with (at least) the
following options:

\begin{lstlisting}
  export TRILINOS_HOME=/path/to/your/Trilinos/source/directory
  cmake \
      -D Trilinos_ENABLE_Amesos:BOOL=ON \
      -D Trilinos_ENABLE_Amesos2:BOOL=ON \
      -D Amesos2_ENABLE_KLU2:BOOL=ON \
      -D Trilinos_ENABLE_AztecOO:BOOL=ON \
      -D Trilinos_ENABLE_Epetra:BOOL=ON \
      -D Trilinos_ENABLE_EpetraExt:BOOL=ON \
      -D Trilinos_ENABLE_Fortran:BOOL=OFF \
      -D Trilinos_ENABLE_Ifpack:BOOL=ON \
      -D Trilinos_ENABLE_Ifpack:BOOL=ON \
      -D Trilinos_ENABLE_MueLu:BOOL=ON \
      -D Trilinos_ENABLE_Teuchos:BOOL=ON \
      -D Trilinos_ENABLE_Tpetra:BOOL=ON \
      -D TPL_ENABLE_MPI:BOOL=OFF \
      -D TPL_ENABLE_MATLAB:BOOL=ON \
      -D MATLAB_ROOT:STRING="<my matlab root>" \
      -D MATLAB_ARCH:STRING="<my matlab os string>" \
  ${TRILINOS_HOME}
\end{lstlisting}

Since \muemex supports both the Epetra and Tpetra linear algebra
libraries, you have to have both enabled in order to build \muemex.
Most additional options can be specified as well.  It is important to
note that \muemex does not work properly with MPI, hence MPI must be
disabled in order to compile \muemex.  The MATLAB\_ARCH option is new to
the cmake build system, and involves the MATLAB-specific architecture
code for your system.  There is currently no automatic way to extract
this, so it must be user-specified.  As of MATLAB 7.9 (R2009b), common
arch codes are:
\begin{center}
\begin{tabular}{l|l}
Code& OS\\
\hline
glnx86& 32-bit Linux (intel/amd)\\
glnxa64& 64-bit Linux (intel/amd)\\
maci64& 64-bit MacOS\\
maci& 32-bit MacOS\\
\end{tabular}
\end{center}

On 64-bit Intel/AMD architectures, Trilinos and all relevant TPLs
(note: this includes BLAS and LAPACK)
must be compiled with the \texttt{-fPIC} option.  This necessitates adding:
\begin{lstlisting}
    -D CMAKE_CXX_FLAGS:STRING="-fPIC" \
    -D CMAKE_C_FLAGS:STRING="-fPIC" \
    -D CMAKE_Fortran_FLAGS:STRING="-fPIC" \
\end{lstlisting}
to the cmake configure line.

\subsection{BLAS \& LAPACK Option \#1: Static Builds}
Trilinos does not play nicely with MATLAB's default LAPACK and BLAS on
64-bit machines.
If \muemex randomly crashes when you run with any Krylov method that
has orthogonalization, chances are \muemex is finding the wrong
BLAS/LAPACK libraries.
This leaves you
with one of two options.  The first is to build them both \textit{statically}
and then specify them as follows:
\begin{lstlisting}
    -D LAPACK_LIBRARY_DIRS:STRING="<path to my lapack.a>" \
    -D BLAS_LIBRARY_DIRS:STRING="<path to my blas.a>" \
\end{lstlisting}
Using static linking for LAPACK and BLAS prevents MATLAB's default libraries to take precedence.

\subsection{BLAS \& LAPACK Option \#2: LD$\_$PRELOAD}
The second option is to use \textsf{LD\_PRELOAD} to tell MATLAB exactly
which libraries to use.  For this option, you can use the dynamic
libraries installed on your system.
Before starting MATLAB, set
LD\_PRELOAD to the paths of libstdc++.so corresponding to the version of GCC used
to build Trilinos, and the paths of libblas.so and liblapack.so on your local system.

For example, if you use bash, you'd do something like this
\begin{lstlisting}
  export LD_PRELOAD=<path>/libstdc++.so:<path>/libblas.so:<path>/liblapack.so
  \end{lstlisting}

For csh / tcsh, do this
\begin{lstlisting}
  setenv LD_PRELOAD <path>/libstdc++.so:<path>/libblas.so:<path>/liblapack.so
\end{lstlisting}

\section{Using \muemex}\label{sec:muemex:usage}
\muemex is designed to be interfaced with via the MATLAB script
\texttt{muelu.m}.  There are five modes in which \muemex can be run:
\begin{enumerate}
\item Setup Mode --- Performs the problem setup for \muelu.
  Depending on whether or not the \texttt{Linear Algebra} option is
  used, \muemex creates either an unpreconditioned Epetra problem,
  an Epetra problem with \muelu, or a Tpetra problem with \muelu.
  The default is \texttt{epetra} for real matrices and \texttt{tpetra}
  for complex matrices. The \texttt{tpetra} option
  supports both real and complex and will infer the scalar type
  from the matrix passed during setup.  This call returns a problem
  handle used to reference the problem in the future, and (optionally)
  the operator complexity, if a preconditioner is being used.
\item Solve Mode --- Given a problem handle and a right-hand side, \muemex
  solves the problem specified.  Setup mode must be called before
  solve mode.
\item Cleanup Mode --- Frees the memory allocated to internal \muelu,
  Epetra and Tpetra objects.  This can be called with a particular
  problem handle, in which case it frees that problem, or without one,
  in which case all \muemex memory is freed.
\item Status Mode --- Prints out status information on problems which
  have been set up.  Like cleanup, it can be called with or without a
  particular problem handle.
\end{enumerate}
All of these modes, with the exception of status and cleanup take
option lists which will be directly converted into
\texttt{Teuchos::ParameterList} objects by \muemex, as key-value pairs.
Options passed during setup will apply to the \muelu preconditioner, and
options passed during a solve will apply to Belos.

\subsection{Setup Mode}
Setup mode is called as follows:
\begin{lstlisting}[language=Matlab]
  >> [h, oc] = muelu('setup', A[, 'parameter', value,...])
\end{lstlisting}
The parameter \texttt{A} represents the sparse matrix to perform aggregation on
and the parameter/value pairs represent standard \muelu options.

The routine returns a problem handle, \texttt{h}, and the operator
complexity \texttt{oc} for the operator.  In addition to the standard
options, setup mode has one unique option of its own:

\choicebox{\tt Linear Algebra}{[{\tt string}] Whether to use
  'epetra unprec', 'epetra', or 'tpetra'. Default is 'epetra' for
  real matrix and 'tpetra' for complex matrix.}

\subsection{Solve Mode}
Solve mode is called as follows:
\begin{lstlisting}[language=Matlab]
  >> [x, its] = muelu(h[, A], b[, 'parameter', value,...])
\end{lstlisting}
The parameter \texttt{h} is a problem handle returned by the
setup mode call, \texttt{A} is the sparse matrix with which to
solve and \texttt{b} is the right-hand side.  Parameter/value pairs
to configure the Belos solver are listed as above. If A is not supplied,
the matrix provided when setting up the problem will be used. \texttt{x} is
the solution multivector with the same dimensions as \texttt{b}, and \texttt{its}
is the number of iterations Belos needed to solve the problem.

All of these options are taken directly from Belos, so consult its
manual for more information. Belos output style and verbosity settings
are implemented as enums, but can be set as strings in \muemex. For example:

\begin{lstlisting}[language=Matlab]
  >> x = muelu(0, b, 'Verbosity', 'Warnings + IterationDetails', ...
                       'Output Style', 'Brief');
\end{lstlisting}

Verbosity settings can be separated by spaces, '+' or ','. Belos::Brief
is the default output style.

\subsection{Cleanup Mode}
Cleanup mode is called as follows:
\begin{lstlisting}[language=Matlab]
  >> muelu('cleanup'[, h])
\end{lstlisting}
The parameter \texttt{h} is a problem handle returned by the
setup mode call and is optional.  If \texttt{h} is provided, that
problem is cleaned up.  If the option is not provided all currently
set up problems are cleaned up.

\subsection{Status Mode}
Status mode is called as follows:
\begin{lstlisting}[language=Matlab]
  >> muelu('status'[, h])
\end{lstlisting}
The parameter \texttt{h} is a problem handle returned by the
setup mode call and is optional.  If \texttt{h} is provided, status
information for that problem is printed.  If the option is not provided all currently
set up problems have status information printed.

\subsection{Tips and Tricks }\label{sec:muemex:tips}

Internally, MATLAB represents all data as doubles unless you go
through efforts to do otherwise.  \muemex detects integer parameters by
a relative error test, seeing if the relative difference between the
value from MATLAB and the value of the \texttt{int}-typecast value are
less than 1e-15.  Unfortunately, this means that \muemex will choose the
incorrect type for parameters which are doubles that happen to have an
integer value (a good example of where this might happen would be the parameter
`smoother Chebyshev: alpha', which defaults to 30.0).  Since \muemex does no
internal typechecking of
parameters (it uses \muelu's internal checks), it has no way of detecting
this conflict.  From the user's perspective, avoiding this is as
simple as adding a small perturbation (greater than a relative 1e-15)
to the parameter that makes it non-integer valued.
