%
% $Id: SANDExampleReportNotstrict.tex,v 1.26 2009-05-01 20:59:19 rolf Exp $
%
% This is an example LaTeX file which uses the SANDreport class file.
% It shows how a SAND report should be formatted, what sections and
% elements it should contain, and how to use the SANDreport class.
% It uses the LaTeX report class, but not the strict option.
%
% Get the latest version of the class file and more at
%    http://www.cs.sandia.gov/~rolf/SANDreport
%
% This file and the SANDreport.cls file are based on information
% contained in "Guide to Preparing {SAND} Reports", Sand98-0730, edited
% by Tamara K. Locke, and the newer "Guide to Preparing SAND Reports and
% Other Communication Products", SAND2002-2068P.
% Please send corrections and suggestions for improvements to
% Rolf Riesen, Org. 9223, MS 1110, rolf@cs.sandia.gov
%
\documentclass[pdf,12pt,report]{SANDreport}
\usepackage{algpseudocode}
\usepackage{amsthm}
\usepackage{booktabs}
\usepackage{calc}
\usepackage{color}
\usepackage{eso-pic}
\usepackage{fancyhdr}
\usepackage{ifthen}
\usepackage{indentfirst}
\usepackage{geometry}
\usepackage{graphicx}
\usepackage[colorlinks, bookmarksopen, %pagebackref=true, backref=page,
             linkcolor={blue},
             anchorcolor={black},
             citecolor={blue},
             filecolor={magenta},
             menucolor={blue},
             pagecolor={red},
             plainpages=false,pdfpagelabels,
             pdfauthor={Jonathan J. Hu, Andrey Prokopenko, Tobias Wiesner, Chris Siefert, Ray Tuminaro},
             pdftitle={MueLu User's Guide},
             pdfkeywords={MueLu,AMG,multigrid,guide,user},
             urlcolor={blue}]{hyperref}
\usepackage{listings}
\usepackage{mathptmx}	% Use the Postscript Times font
\usepackage{multirow}
\usepackage{pifont}
\usepackage[FIGBOTCAP,normal,bf,tight]{subfigure}
\usepackage{tabularx}
\usepackage{verbatim}
\usepackage{xspace}
\usepackage{flowchart} % also loads tikz
\usepackage{algorithm}
\usetikzlibrary{arrows}

%\usepackage{draftwatermark}
%\SetWatermarkScale{.5}

\algrenewcommand{\algorithmiccomment}[1]{\hskip3em // #1}


% If you want to relax some of the SAND98-0730 requirements, use the "relax"
% option. It adds spaces and boldface in the table of contents, and does not
% force the page layout sizes.
% e.g. \documentclass[relax,12pt]{SANDreport}
%
% You can also use the "strict" option, which applies even more of the
% SAND98-0730 guidelines. It gets rid of section numbers which are often
% useful; e.g. \documentclass[strict]{SANDreport}



% ---------------------------------------------------------------------------- %
%
% Set the title, author, and date
%
\title{MueLu User's Guide 1.0 \\
(Trilinos version 11.12)}

\author{Andrey Prokopenko \\
  Scalable Algorithms \\
  Sandia National Laboratories\\
  Mailstop 1318 \\
  P.O.~Box 5800 \\
  Albuquerque, NM 87185-1318\\
  aprokop@sandia.gov\\
  \and
  Tobias Wiesner \\
  Institute for Computational Mechanics \\
  Technische Universit\"at M\"unchen\\
  Boltzmanstra\ss e 15 \\
  85747 Garching, Germany\\
  wiesner@lnm.mw.tum.de\\
  \and
  Jonathan J. Hu \\
  Scalable Algorithms \\
  Sandia National Laboratories\\
  Mailstop 9159 \\
  P.O.~Box 0969 \\
  Livermore, CA 94551-0969\\
  jhu@sandia.gov
  \and
  Christopher M. Siefert\\
  Computational Multiphysics\\
  Sandia National Laboratories\\
  Mailstop 1322 \\
  P.O.~Box 5800 \\
  Albuquerque, NM 87185-1322\\
  csiefer@sandia.gov
  \and
  Raymond S. Tuminaro\\
  Computational Mathematics\\
  Sandia National Laboratories\\
  Mailstop 9159 \\
  P.O.~Box 0969 \\
  Livermore, CA 94551-0969\\
  rstumin@sandia.gov
}

% There is a "Printed" date on the title page of a SAND report, so
% the generic \date should generally be empty.
\date{}

\newcommand{\amesos}       {\textsc{Amesos}\xspace}
\newcommand{\amesostwo}    {\textsc{Amesos2}\xspace}
\newcommand{\anasazi}      {\textsc{Anasazi}\xspace}
\newcommand{\aztecoo}      {\textsc{AztecOO}\xspace}
\newcommand{\belos}        {\textsc{Belos}\xspace}
\newcommand{\epetra}       {\textsc{Epetra}\xspace}
\newcommand{\epetraext}    {\textsc{EpetraExt}\xspace}
\newcommand{\galeri}       {\textsc{Galeri}\xspace}
\newcommand{\ifpack}       {\textsc{Ifpack}\xspace}
\newcommand{\ifpacktwo}    {\textsc{Ifpack2}\xspace}
\newcommand{\isorropia}    {\textsc{Isorropia}\xspace}
\newcommand{\loca}         {\textsc{Loca}\xspace}
\newcommand{\ml}           {\textsc{ML}\xspace}
\newcommand{\muelu}        {\textsc{MueLu}\xspace}
\newcommand{\nox}          {\textsc{NOX}\xspace}
\newcommand{\stratimikos}  {\textsc{Stratimikos}\xspace}
\newcommand{\teuchos}      {\textsc{Teuchos}\xspace}
\newcommand{\tpetra}       {\textsc{Tpetra}\xspace}
\newcommand{\trilinos}     {\textsc{Trilinos}\xspace}
\newcommand{\xpetra}       {\textsc{Xpetra}\xspace}
\newcommand{\zoltan}       {\textsc{Zoltan}\xspace}
\newcommand{\zoltantwo}    {\textsc{Zoltan2}\xspace}


\newcommand{\klu}          {\textsc{Klu}\xspace}
\newcommand{\metis}        {\textsc{Metis}\xspace}
\newcommand{\mumps}        {\textsc{Mumps}\xspace}
\newcommand{\umfpack}      {\textsc{Umfpack}\xspace}
\newcommand{\superlu}      {\textsc{SuperLU}\xspace}
\newcommand{\superludist}  {\textsc{SuperLU\_dist}\xspace}
\newcommand{\parmetis}     {\textsc{ParMetis}\xspace}
\newcommand{\paraview}     {\textsc{ParaView}\xspace}

\newcommand{\parameterlist}{\texttt{ParameterList}\xspace}

\newcommand \trilinosWeb   {trilinos.sandia.gov\xspace}

%\newcommand{\be}  {\begin{enumerate}}
%\newcommand{\ee}  {\end{enumerate}}
%\newcommand{\cba}[3]{\choicebox{\texttt{#1}}{[{\texttt #2}] #3}}
%\newcommand{\cbb}[4]{\choicebox{\texttt{#1}}{[{\texttt #2}] #4 {\bf Default:~}#3.}}
%\newcommand{\cbc}[4]{\choicebox{\texttt{\color{red}#1}}{[{\texttt #2}] #4 {\bf Default:~}#3.}}
%
%\newcommand{\comm}[2]{\bigskip
%                      \begin{tabular}{|p{4.5in}|}\hline
%                      \multicolumn{1}{|c|}{{\bf Comment by #1}}\\ \hline
%                      #2\\ \hline
%                      \end{tabular}\\
%                      \bigskip
%                     }


\newtheorem*{mycomment}{\ding{42}}
\newtheoremstyle{plain}
  {\topsep}   % ABOVESPACE
  {\topsep}   % BELOWSPACE
  {\normalfont}  % BODYFONT
  {0pt}       % INDENT (empty value is the same as 0pt)
  {\bfseries} % HEADFONT
  {}         % HEADPUNCT
  {5pt plus 1pt minus 1pt} % HEADSPACE
  {}          % CUSTOM-HEAD-SPEC

% further declarations and additional commands
\definecolor{hellgelb}{rgb}{1,1,0.8}   % background color for C++ listings
\definecolor{darkgreen}{rgb}{0.0, 0.2, 0.13}
%\definecolor{hellrot}{HTML}{FFA4C2}    % background color for xml files

% settings for listings package
\lstset{
  backgroundcolor=\color{hellgelb},
  basicstyle=\ttfamily\small,
  breakautoindent=true,
  breaklines=true,
  captionpos=b,
  columns=flexible,
  commentstyle=\color{darkgreen},
  extendedchars=true,
  float=hbp,
  frame=single,
  identifierstyle=\color{black},
  keywordstyle=\color{blue},
  numbers=none,
  numberstyle=\tiny,
  showspaces=false,
  showstringspaces=false,
  stringstyle=\color{purple},
  tabsize=2,
}


% ---------------------------------------------------------------------------- %
% Set some things we need for SAND reports. These are mandatory
%
\SANDnum{SAND2014-18874}
\SANDprintDate{October 2014}
\SANDauthor{Andrey Prokopenko, Jonathan J. Hu, Tobias A. Wiesner, Christopher M. Siefert, Raymond S. Tuminaro}


% ---------------------------------------------------------------------------- %
% Include the markings required for your SAND report. The default is "Unlimited
% Release". You may have to edit the file included here, or create your own
% (see the examples provided).
%
% \include{MarkUR} % Not needed for unlimted release reports


% ---------------------------------------------------------------------------- %
% The following definition does not have a default value and will not
% print anything, if not defined
%
%\SANDsupersed{SAND1901-0001}{January 1901}
%\input{MarkOUO}


% ---------------------------------------------------------------------------- %
%
% Start the document
%
\begin{document}
    \maketitle

    % ------------------------------------------------------------------------ %
    % An Abstract is required for SAND reports
    %
    \begin{abstract}
	%This is the definitive user guide for the \muelu{} library in Trilinos version XX.YY.
%\muelu{} is a C++ multigrid framework that can work with either the \epetra or \tpetra linear
%algebra libraries.
%\muelu{} provides a variety of aggregation-based multigrid algorithms,
%including smoothed aggregation algebraic multigrid (AMG), Petrov-Galerkin AMG, and AMG for
%Maxwell's equations, as well as many different types of smoothers.
%\muelu{} is templated on the index, scalar, and compute node types.
%Thus it is possible to use \muelu{} on problems with scalar types other than double, on very
%large problems, and to exploit node-level parallelism.

This is the official user guide for \muelu{} multigrid library in Trilinos
version 11.12.  This guide provides an overview of \muelu, its capabilities, and
instructions for new users who want to start using \muelu{} with a minimum of
effort. Detailed information is given on how to drive \muelu{} through its XML
interface. Links to more advanced use cases are given. This guide gives
information on how to achieve good parallel performance, as well as how to
introduce new algorithms. Finally, readers will find a comprehensive listing of
available \muelu{} options.  {\em Any options not documented in this manual
should be considered strictly experimental.}

    \end{abstract}


    % ------------------------------------------------------------------------ %
    % An Acknowledgement section is optional but important, if someone made
    % contributions or helped beyond the normal part of a work assignment.
    % Use \section* since we don't want it in the table of context
    %
    \clearpage
    \chapter*{Acknowledgment}
	Many people have helped develop \muelu{} and/or provided valuable feedback, and
we would like to acknowledge their contributions here: Tom Benson, Julian
Cortial, Eric Cyr, Stefan Domino, Travis Fisher, Jeremie Gaidamour, Axel
Gerstenberger, Chetan Jhurani, Mark Hoemmen, Paul Lin, Eric Phipps, Siva
Rajamanickam, Nico Schl{\"o}mer, and Paul Tsuji.



    % ------------------------------------------------------------------------ %
    % The table of contents and list of figures and tables
    % Comment out \listoffigures and \listoftables if there are no
    % figures or tables. Make sure this starts on an odd numbered page
    %
    \cleardoublepage		% TOC needs to start on an odd page
    \tableofcontents
    \listoffigures
    \listoftables


    % ---------------------------------------------------------------------- %
    % An optional preface or Foreword
    %\clearpage
    %\chapter*{Preface}
    %\addcontentsline{toc}{chapter}{Preface}
	%\input{CommonPreface}


    % ---------------------------------------------------------------------- %
    % An optional executive summary
    %\clearpage
    %\chapter*{Summary}
    %\addcontentsline{toc}{chapter}{Summary}
	%\input{CommonSummary}


    % ---------------------------------------------------------------------- %
    % An optional glossary. We don't want it to be numbered
    %\clearpage
    %\chapter*{Nomenclature}
    %\addcontentsline{toc}{chapter}{Nomenclature}
    %\begin{description}
	%\item[dry spell]
	%    using a dry erase marker to spell words
	%\item[dry wall]
	%    the writing on the wall
	%\item[dry humor]
	%    when people just do not understand
	%\item[DRY]
	%    Don't Repeat Yourself
    %\end{description}


    % ---------------------------------------------------------------------- %
    % This is where the body of the report begins; usually with an Introduction
    %
    \SANDmain		% Start the main part of the report

    %-----------------------------%
    \chapter{Introduction}\label{sec:introduction}
    %-----------------------------%
    This guide gives an overview of \muelu{}'s capabilities.  If you are looking for
a tutorial, please refer to the \muelu{} tutorial in \verb!muelu/doc/Tutorial!
(see also~\cite{MueLuTutorial}). New users should start with~\S\ref{sec:getting
started}. It strives to give the new user all the information he/she might need
to begin using \muelu{} quickly. Users interested in performance, especially in
parallel context, should refer to~\S\ref{sec:performance}.  Users looking for a
particular option should consult~\S\ref{sec:options}, containing a complete set of
supported options in \muelu{}. \\

\noindent
If you find any errors or omissions in this guide, have comments or suggestions,
or would like to contribute to \muelu{} development, please contact the \muelu{}
\href{mailto:muelu-users@software.sandia.gov}{users list}, or
\href{mailto:muelu-developers@software.sandia.gov}{developers list}.


    %-----------------------------%
    \chapter{Multigrid background}\label{sec:multigrid}
    %-----------------------------%
    \label{sec:multigrid intro}
Here we provide a brief multigrid introduction (see~\cite{MGTutorial}
or~\cite{OwlBook} for more information). A multigrid solver tries to approximate
the original problem of interest with a sequence of smaller (\textit{coarser})
problems. The solutions from the coarser problems are combined in order to
accelerate convergence of the original (\textit{fine}) problem on the finest
grid. A simple multilevel iteration is illustrated in
Algorithm~\ref{multigrid_code}.

\begin{algorithm}
\centering
\begin{algorithmic}[0]
  \State{$A_0 = A$}
  \Function{Multilevel}{$A_k$, $b$, $u$, $k$}
    \State{// Solve $A_k$ u = b (k is current grid level)}
    \State $ u = S^{1}_m (A_k, b, u)$
      \If{$(k \ne {\bf N-1})$}
        \State{$P_k = $ determine\_interpolant( $A_k$ )}
        \State{$R_k = $ determine\_restrictor( $A_k$ )}
        \State{$\widehat{r}_{k+1} = R_k (b - A_k u )$}
        \State{$A_{k+1} = R_k A_k P_k$}
        \State{$v = 0$}
        \State{}\Call{Multilevel}{$\widehat{A}_{k+1}$, $\widehat{r}_{k+1}$, $v$, $k+1$}
        \State{$ u = u + P_{k} v$}
        \State{$ u = S^{2}_m (A_k, b, u )$}
      \EndIf
  \EndFunction
\end{algorithmic}
\caption{V-cycle multigrid with $N$ levels to solve $Ax=b$.}
\label{multigrid_code}
\end{algorithm}

In the multigrid iteration in Algorithm~\ref{multigrid_code}, the $S^{1}_m()$'s
and $S^{2}_m()$'s are called \textit{pre-smoothers} and \textit{post-smoothers}.
They are approximate solvers (e.g., symmetric Gauss-Seidel), with the subscript
$m$ denoting the number of applications of the approximate solution method. The
purpose of a smoother is to quickly reduce certain error modes in the
approximate solution on a level $k$. For symmetric problems, the pre-
and post-smoothers should be chosen to maintain symmetry (e.g., forward
Gauss-Seidel for the pre-smoother and backward Gauss-Seidel for the
post-smoother). The $P_k$'s are \textit{interpolation} matrices that transfer
solutions from coarse levels to finer levels. The $R_k$'s are
\textit{restriction} matrices that restrict a fine level solution to a coarser
level. In a geometric multigrid, $P_k$'s and $R_k$'s are determined
by the application, whereas in an algebraic multigrid they are automatically
generated. For symmetric problems, typically $R_k=P_k^T$. For nonsymmetric
problems, this is not necessarily true. The $A_k$'s are the coarse level
problems, and are generated through a Galerkin (triple matrix) product.

Please note that the algebraic multigrid algorithms implemented in \muelu{}
generate the grid transfers $P_k$ automatically and the coarse problems $A_k$
via a sparse triple matrix product. \trilinos{} provides a wide selection of
smoothers and direct solvers for use in \muelu through the \ifpack,
\ifpacktwo, \amesos, and \amesostwo packages (see \S\ref{sec:options}).



    %-----------------------------%
    \chapter{Getting Started}\label{sec:getting started}
    %-----------------------------%
    This section is meant to get you using \muelu{} as quickly as possible.  \S\ref{sec:overview} gives a
summary of \muelu's design.  \S\ref{sec:configuration and build} lists \muelu's dependencies on other
\trilinos libraries and provides a sample cmake configuration line.  Finally, code examples using the XML
interface are given in \S\ref{sec:examples in code}.

\label{sec:overview}
\section{Overview of \muelu}
%algorithm types
%problems types
\muelu{} is an extensible algebraic multigrid (AMG) library that is part of the
\trilinos{} project. \muelu{} works with \epetra (32-bit version
\footnote{Support for the Epetra 64-bit version is planned.}) and
\tpetra matrix types. The library is written in C++ and allows for different
ordinal (index) and scalar types.  \muelu{} is designed to be efficient on many
different computer architectures, from workstations to supercomputers, relying
on ``MPI+X" principle, where ``X" can be threading or CUDA.

\muelu{} provides a number of different multigrid algorithms:
\be
  \item smoothed aggregation AMG (for Poisson-like and elasticity problems);
  \item Petrov-Galerkin aggregation AMG (for convection-diffusion problems);
  \item energy-minimizing AMG;
  \item aggregation-based AMG for problems arising from the eddy current
    formulation of Maxwell's equations.
\ee
\muelu's software design allows for the rapid introduction of new multigrid algorithms.
The most important features of \muelu{} can be summarized as:
\begin{description}
  \item \textbf{Easy-to-use interface}

    \muelu{} has a user-friendly parameter input deck which covers
    most important use cases.  Reasonable defaults are provided for common problem types
    (see Table \ref{t:problem_types}).

  \item \textbf{Modern object-oriented software architecture}

    \muelu{} is written completely in C++ as a modular object-oriented multigrid
    framework, which provides flexibility to combine and reuse existing
    components to develop novel multigrid methods.

  \item \textbf{Extensibility}

    Due to its flexible design, \muelu{} is an excellent toolkit for
    research on novel multigrid concepts. Experienced multigrid users have full
    access to the underlying framework through an advanced XML based interface.
    Expert users may use and extend the C++ API directly.

  \item \textbf{Integration with \trilinos{} library}

    As a package of \trilinos, \muelu{} is well integrated into the \trilinos
    environment. \muelu{} can be used with either the \tpetra{} or \epetra{}
    (32-bit) linear algebra stack. It is templated on the local index, global
    index, scalar, and compute node types. This makes \muelu{} ready for
    future developments.

  \item \textbf{Broad range of supported platforms}

    \muelu{} runs on wide variety of architectures, from desktop workstations to
    parallel Linux clusters and supercomputers (~\cite{lin2014}).

  \item \textbf{Open source}

    \muelu{} is freely available through a simplified BSD license (see Appendix~\ref{sec:license}).
\end{description}

\section{Configuration and Build}
\label{sec:configuration and build}

\muelu{} has been compiled successfully under Linux with the following C++
compilers: GNU versions 4.1 and later, Intel versions 12.1/13.1, and clang versions 3.2 and later.
In the future, we recommend using compilers supporting C++11 standard.

\subsection{Dependencies}

\noindent{\bf Required Dependencies}

\muelu{} requires that \teuchos{} and either \epetra/\ifpack or \tpetra/\ifpacktwo
are enabled.

\noindent{\bf Recommended Dependencies}

We strongly recommend that you enable the following \trilinos libraries along with \muelu:

\begin{itemize}
  \item \epetra stack: \aztecoo, \epetra, \amesos, \ifpack, \isorropia, \galeri,
    \zoltan;
  \item \tpetra stack: \amesostwo, \belos, \galeri, \ifpacktwo, \tpetra,
    \zoltantwo.
\end{itemize}

\noindent{\bf Tutorial Dependencies}

In order to run the \muelu{} Tutorial \cite{MueLuTutorial} located in \verb!muelu/doc/Tutorial!, \muelu{} must be configured with the following
dependencies enabled:

  \aztecoo, \amesos, \amesostwo, \belos, \epetra, \ifpack, \ifpacktwo, \isorropia,
  \galeri, \tpetra, \zoltan, \zoltantwo.

\begin{mycomment}
Note that the \muelu{} tutorial \cite{MueLuTutorial} comes with a VirtualBox image with a pre-installed
Linux and \trilinos{}.   In this way, a user can immediately begin experimenting with \muelu{} without
having to install the \trilinos{} libraries. Therefore, it is an ideal starting point to get in touch with \muelu{}.
\end{mycomment}

\noindent{\bf Complete List of Direct Dependencies}

\begin{table}[ht]
  \centering
  \begin{tabular}{p{3.5cm} c c c c}
    \toprule
    \multirow{2}{*}{Dependency} & \multicolumn{2}{c}{Required} & \multicolumn{2}{c}{Optional} \\
    \cmidrule(r){2-3} \cmidrule(l){4-5}
                   & Library  & Testing  & Library  & Testing        \\
    \hline
    \amesos        &          &          & $\times$ & $\times$  \\
    \amesostwo     &          &          & $\times$ & $\times$  \\
    \aztecoo       &          &          &          & $\times$  \\
    \belos         &          &          &          & $\times$  \\
    \epetra        &          &          & $\times$ & $\times$  \\
    \ifpack        &          &          & $\times$ & $\times$  \\
    \ifpacktwo     &          &          & $\times$ & $\times$  \\
    \isorropia     &          &          & $\times$ & $\times$  \\
    \galeri        &          &          &          & $\times$  \\
    \kokkosclassic &          &          & $\times$ & \\
    \teuchos{}     & $\times$ & $\times$ &          & \\
    \tpetra        &          &          & $\times$ & $\times$  \\
    \xpetra        & $\times$ & $\times$ &          & \\
    \zoltan        &          &          & $\times$ & $\times$  \\
    \zoltantwo     &          &          & $\times$ & $\times$  \\
    \midrule
    Boost          &          &          & $\times$ & \\
    BLAS           & $\times$ & $\times$ &          & \\
    LAPACK         & $\times$ & $\times$ &          & \\
    MPI            &          &          & $\times$ & $\times$  \\
    \bottomrule
  \end{tabular}
  \caption{\label{tab:dependencies}\muelu's required and optional dependencies,
    subdivided by whether a dependency is that of the \muelu{} library itself
    (\textit{Library}), or of some \muelu{} test (\textit{Testing}). }
\end{table}

Table~\ref{tab:dependencies} lists the dependencies of \muelu, both required and
optional. If an optional dependency is not present, the tests requiring that
dependency are not built.

\begin{mycomment}
\amesos{}/\amesostwo{} are necessary if one wants to use a sparse direct solve on the coarsest level.
\zoltan{}/\zoltantwo{} are necessary if one wants to use matrix rebalancing in parallel runs (see~\S\ref{sec:performance}).
\aztecoo{}/\belos{} are necessary if one wants to test \muelu{} as a preconditioner instead of a solver.
\end{mycomment}

\begin{mycomment}
\muelu{} has also been successfully tested with SuperLU 4.1 and SuperLU 4.2.
\end{mycomment}
\begin{mycomment}
Some packages that \muelu{} depends on may come with additional requirements for
third party libraries, which are not listed here as explicit dependencies of \muelu{}.
It is highly recommended to install ParMetis 3.1.1 or newer for \zoltan{},
ParMetis 4.0.x for \zoltantwo{}, and SuperLU 4.1 or newer for
\amesos{}/\amesostwo{}.
\end{mycomment}

\subsection{Configuration}
The preferred way to configure and build \muelu{} is to do that outside of the source directory.
Here we provide a sample configure script that will enable \muelu{} and all of its optional dependencies:
\begin{lstlisting}
  export TRILINOS_HOME=/path/to/your/Trilinos/source/directory
  cmake \
      -D BUILD_SHARED_LIBS:BOOL=ON \
      -D CMAKE_BUILD_TYPE:STRING="RELEASE" \
      -D CMAKE_CXX_FLAGS:STRING="-g" \
      -D Trilinos_ENABLE_EXPLICIT_INSTANTIATION:BOOL=ON \
      -D Trilinos_ENABLE_TESTS:BOOL=OFF \
      -D Trilinos_ENABLE_EXAMPLES:BOOL=OFF \
      -D Trilinos_ENABLE_MueLu:BOOL=ON \
      -D MueLu_ENABLE_TESTS:STRING=ON \
      -D MueLu_ENABLE_EXAMPLES:STRING=ON \
      -D TPL_ENABLE_BLAS:BOOL=ON \
      -D TPL_ENABLE_MPI:BOOL=ON \
  ${TRILINOS_HOME}
\end{lstlisting}

\noindent
More configure examples can be found in \texttt{Trilinos/sampleScripts}.
For more information on configuring, see the \trilinos Cmake Quickstart guide \cite{TrilinosCmakeQuickStart}.

\section{Examples in code}
\label{sec:examples in code}
% simple scaling test
%   galeri
%   XML input
%   belos/aztecoo or stand-alone solver
%   look @ tutorial or elsewhere for more advanced usage

The most commonly used scenario involving \muelu{} is using a multigrid
preconditioner inside an iterative linear solver. In \trilinos{}, a user has a
choice between \epetra and \tpetra for the underlying linear algebra library.
Important Krylov subspace methods (such as preconditioned CG and GMRES) are
provided in \trilinos{} packages \aztecoo (\epetra{}) and \belos
(\epetra{}/\tpetra{}).

At this point, we assume that the reader is comfortable with \teuchos{} referenced-counted
pointers (RCPs) for memory management (an introduction to RCPs can be found
in~\cite{RCP2010}) and the \texttt{Teuchos::ParameterList} class~\cite{TeuchosURL}.

\subsection{\muelu{} as a preconditioner within \belos}
\label{sec:tpetraexample}
The following code shows the basic steps of how to use a \muelu{}
multigrid preconditioner with \tpetra{} linear algebra library and with a linear
solver from \belos{}. To keep the example short and clear, we skip the template
parameters and focus on the algorithmic outline of setting up
a linear solver. For further details, a user may refer to the \texttt{examples} and
\texttt{test} directories.

First, we create the \muelu{} multigrid preconditioner. It can be done in a few
ways. For instance, multigrid parameters can be read from an XML file
(e.g., \textit{mueluOptions.xml} in the example below).
\begin{lstlisting}[language=C++]
    Teuchos::RCP<Tpetra::CrsMatrix<> > A;
    // create A here ...
    std::string optionsFile = "mueluOptions.xml";
    Teuchos::RCP<MueLu::TpetraOperator> mueLuPreconditioner =
       MueLu::CreateTpetraPreconditioner(A, optionsFile);
\end{lstlisting}
The XML file contains multigrid options. A typical file with \muelu{} parameters
looks like the following.
\begin{lstlisting}[language=XML]
<ParameterList name="MueLu">

  <Parameter name="verbosity" type="string" value="low"/>

  <Parameter name="max levels" type="int" value="3"/>
  <Parameter name="coarse: max size" type="int" value="10"/>

  <Parameter name="multigrid algorithm" type="string" value="sa"/>

  <!-- Damped Jacobi smoothing -->
  <Parameter name="smoother: type" type="string" value="RELAXATION"/>
  <ParameterList name="smoother: params">
    <Parameter name="relaxation: type"  type="string" value="Jacobi"/>
    <Parameter name="relaxation: sweeps" type="int" value="1"/>
    <Parameter name="relaxation: damping factor" type="double" value="0.9"/>
  </ParameterList>

  <!-- Aggregation -->
  <Parameter name="aggregation: type" type="string" value="uncoupled"/>
  <Parameter name="aggregation: min agg size" type="int" value="3"/>
  <Parameter name="aggregation: max agg size" type="int" value="9"/>

</ParameterList>
\end{lstlisting}
It defines a three level smoothed aggregation multigrid algorithm. The
aggregation size is between three and nine(2D)/27(3D) nodes.  One sweep with a
damped Jacobi method is used as a level smoother. By default, a direct solver is
applied on the coarsest level. A complete list of available parameters and valid
parameter choices is given in \S\ref{sec:muelu_options} of this User's Guide.

Users can also construct a multigrid preconditioner using a provided \parameterlist
without accessing any files in the following manner.
\begin{lstlisting}[language=C++]
  Teuchos::RCP<Tpetra::CrsMatrix<> > A;
  // create A here ...
  Teuchos::ParameterList paramList;
  paramList.set("verbosity", "low");
  paramList.set("max levels", 3);
  paramList.set("coarse: max size", 10);
  paramList.set("multigrid algorithm", "sa");
  // ...
  Teuchos::RCP<MueLu::TpetraOperator> mueLuPreconditioner =
     MueLu::CreateTpetraPreconditioner(A, paramList);
\end{lstlisting}

Besides the linear operator $A$, we also need an initial guess vector for the
solution $X$ and a right hand side vector $B$ for solving a linear system.
\begin{lstlisting}[language=C++]
    Teuchos::RCP<const Tpetra::Map<> > map = A->getDomainMap();

    // Create initial vectors
    Teuchos::RCP<Tpetra::MultiVector<> > B, X;
    X = Teuchos::rcp( new Tpetra::MultiVector<>(map,numrhs) );
    Belos::MultiVecTraits<>::MvRandom( *X );
    B = Teuchos::rcp( new Tpetra::MultiVector<>(map,numrhs) );
    Belos::OperatorTraits<>::Apply( *A, *X, *B );
    Belos::MultiVecTraits<>::MvInit( *X, 0.0 );
\end{lstlisting}
To generate a dummy example, the above code first declares two vectors. Then, a
right hand side vector is calculated as the matrix-vector product of a random vector
with the operator $A$. Finally, an initial guess is initialized with zeros.

Then, one can define a \texttt{Belos::LinearProblem} object where the
\texttt{mueLuPreconditioner} is used for left preconditioning
\begin{lstlisting}[language=C++]
    Belos::LinearProblem<> problem( A, X, B );
    problem->setLeftPrec(mueLuPreconditioner);
    bool set = problem.setProblem();
\end{lstlisting}

Next, we set up a \belos{} solver using some basic parameters
\begin{lstlisting}[language=C++]
    Teuchos::ParameterList belosList;
    belosList.set( "Block Size", 1 );
    belosList.set( "Maximum Iterations", 100 );
    belosList.set( "Convergence Tolerance", 1e-10 );
    belosList.set( "Output Frequency", 1 );
    belosList.set( "Verbosity", Belos::TimingDetails + Belos::FinalSummary );

    Belos::BlockCGSolMgr<> solver( rcp(&problem,false), rcp(&belosList,false) );
\end{lstlisting}

Finally, we solve the system.
\begin{lstlisting}[language=C++]
    Belos::ReturnType ret = solver.solve();
\end{lstlisting}

\subsection{\muelu{} as a preconditioner for \aztecoo}

For \epetra, users have two library options: \belos{} (recommended) and \aztecoo{}.
\aztecoo{} and \belos both provide fast and mature implementations of common iterative Krylov linear solvers.
\belos has additional capabilities, such as Krylov subspace recycling and ``tall skinny QR".

Constructing a \muelu{} preconditioner for Epetra operators is done in a similar
manner to Tpetra.
\begin{lstlisting}[language=C++]
    Teuchos::RCP<Epetra_CrsMatrix> A;
    // create A here ...
    Teuchos::RCP<MueLu::EpetraOperator> mueLuPreconditioner;
    std::string optionsFile = "mueluOptions.xml";
    mueLuPreconditioner = MueLu::CreateEpetraPreconditioner(A, optionsFile);
\end{lstlisting}
\muelu{} parameters are generally Epetra/Tpetra agnostic, thus the XML parameter file
could be the same as~\S\ref{sec:tpetraexample}.

Furthermore, we assume that a right hand side vector and a solution vector with
the initial guess are defined.
\begin{lstlisting}[language=C++]
    Teuchos::RCP<const Epetra_Map> map = A->DomainMap();
    Teuchos::RCP<Epetra_Vector> B = Teuchos::rcp(new Epetra_Vector(map));
    Teuchos::RCP<Epetra_Vector> X = Teuchos::rcp(new Epetra_Vector(map));
    X->PutScalar(0.0);
\end{lstlisting}

Then, an \texttt{Epetra\_LinearProblem} can be defined.
\begin{lstlisting}[language=C++]
    Epetra_LinearProblem epetraProblem(A.get(), X.get(), B.get());
\end{lstlisting}

The following code constructs an \aztecoo{} CG solver.
\begin{lstlisting}[language=C++]
    AztecOO aztecSolver(epetraProblem);
    aztecSolver.SetAztecOption(AZ_solver, AZ_cg);
    aztecSolver.SetPrecOperator(mueLuPreconditioner.get());
\end{lstlisting}

Finally, the linear system is solved.
\begin{lstlisting}[language=C++]
    int maxIts = 100;
    double tol = 1e-10;
    aztecSolver.Iterate(maxIts, tol);
\end{lstlisting}

\subsection{Further remarks}

This section is only meant to give a brief introduction on how to use \muelu{}
as a preconditioner within the \trilinos{} packages for iterative solvers. There
are other, more complicated, ways to use \muelu{} as a preconditioner for \belos
and \aztecoo through the \xpetra interface. Of course, \muelu{} can also work as
standalone multigrid solver. For more information on these topics, the reader
may refer to the examples and tests in the \muelu{} source directory
(\texttt{Trilinos/packages/muelu}), as well as to the \muelu{} tutorial~\cite{MueLuTutorial}.
For in-depth details of \muelu applied to multiphysics problems, please see~\cite{Wiesner2014}.


    %-----------------------------%
    \chapter{Performance tips}\label{sec:performance}
    %-----------------------------%
    In practice, it can be very challenging to find an appropriate set of multigrid
parameters for a specific problem, especially if few details are known about the
underlying linear system. In this Chapter, we provide some advice for improving
multigrid performance.

\begin{mycomment}
For optimizing multigrid parameters, it is highly recommended to set the
verbosity to \verb|high| or \verb|extreme| for \muelu{} to output more
information (e.g., for the effect of the chosen parameters to the aggregation
and coarsening process).
\end{mycomment}

Some general advice:
\begin{itemize}
  \item
    Choose appropriate iterative linear solver (e.g., GMRES for non-symmetric problems).

  \item
    Start with the recommended settings for particular problem types. See
    Table~\ref{t:problem_types}.

  \item
    Choose reasonable basic multigrid parameters
    (see~\S\ref{sec:options_general}), including maximum number of multigrid
    levels (\texttt{max levels}) and maximum allowed coarse size of the problem
    (\texttt{coarse: max size}). Take fine level problem size and sparsity
    pattern into account for a reasonable choice of these parameters.

  \item
    Select an appropriate transfer operator strategy
    (see~\S\ref{sec:options_mg}). For symmetric problems, you should start with smoothed
    aggregation multigrid. For non-symmetric problems, a Petrov-Galerkin smoothed
    aggregation method is a good starting point, though smoothed aggregation may
    also perform well.

  \item
    Enable implicit restrictor construction (\texttt{transpose:} \texttt{use implicit}) for symmetric
    problems.

  \item
    Find good level smoothers (see~\S\ref{sec:options_smoothing}). If a problem
    is symmetric positive definite, choose a smoother with a matrix-vector
    computational kernel, such as the Chebyshev polynomial smoother. If you are
    using relaxation smoothers, we recommend starting with optimizing the
    damping parameter. Once you have found a good damping parameter for your
    problem, you can increase the number of smoothing iterations.

  \item
    Adjust aggregation parameters if you experience bad coarsening ratios
    (see~\S\ref{sec:options_aggregation}). Particularly, try adjusting the
    minimum (\texttt{aggregation:} \texttt{min agg size}) and maximum
    (\texttt{aggregation:} \texttt{max agg size}) aggregation parameters. For a
    2D (3D) isotropic problem on a regular mesh, the aggregate size should be
    about 9 (27) nodes per aggregate.

  \item
    Replace a direct solver with an iterative method (\texttt{coarse: type}) if
    your coarse level solution becomes too expensive (see~\S\ref{sec:options_smoothing}).
\end{itemize}

Some advice for parallel runs include:
\begin{enumerate}
  \item
    Enable matrix rebalancing when running in parallel (\texttt{repartition:}
    \texttt{enable}).

  \item
    Use smoothers invariant to the number of processors, such as
    polynomial smoothing, if possible.

  \item
    Use \texttt{uncoupled} aggregation instead of \texttt{coupled}, as later
    requires significantly more communication.

  \item
    Adjust rebalancing parameters (see~\S\ref{sec:options_rebalancing}). Try
    choosing rebalancing parameters so that you end up with one processor on the
    coarsest level for the direct solver (this avoids additional communication).

  \item
    Enable implicit rebalancing of prolongators and restrictors
    (\texttt{repartition: rebalance P and R}).
\end{enumerate}


    %-----------------------------%
    \chapter{\muelu{} options} \label{sec:options}
    %-----------------------------%
    \label{sec:muelu_options}

In this section, we report the complete list of \muelu{} input parameters.  It
is important to notice, however, that \muelu{} relies on other \trilinos{}
packages to provide support for some of its algorithms. For instance,
\ifpack{}/\ifpacktwo{} provide standard smoothers like Jacobi, Gauss-Seidel or
Chebyshev, while \amesos{}/\amesostwo{} provide access to direct solvers. The
parameters affecting the behavior of the algorithms in those packages are
simply passed by \muelu{} to a routine from the corresponding library. Please
consult corresponding packages' documentation for a full list of supported
algorithms and corresponding parameters.

\section{Using parameters on individual levels}
Some of the parameters that affect the preconditioner can in principle be
different from level to level. By default, parameters affect all levels in
a multigrid hierarchy.

The settings on a particular levels can be changed by using level sublists.
A level sublist is a \parameterlist{} sublist with the name ``level XX'', where XX is the level number. The
parameter names in the sublist do not require any modifications. For example,
the following fragment of code
\begin{lstlisting}[language=XML]
  <ParameterList name="level 2">
    <Parameter name="smoother: type" type="string" value="CHEBYSHEV"/>
  </ParameterList>
\end{lstlisting}
changes the smoother for level 2 to be a polynomial smoother.

\section{Parameter validation}
By default, \muelu{} validates the input parameter list. A parameter that is
misspelled, is unknown, or has an incorrect value type will cause an exception to be
thrown and execution to halt.

\begin{mycomment}
Spaces are important within a parameter's name. Please separate words
by just one space, and make sure there are no leading or trailing spaces.
\end{mycomment}

The option \verb|print initial parameters| prints the initial list given to the
interpreter. The option \verb|print unused parameters| prints the list of unused
parameters.

% ==================== GENERAL ====================
\section{General options}
\label{sec:options_general}

\begin{table}[h!]
  \begin{center}
    \begin{tabular}{p{3cm} p{12cm}}
      \toprule
      Verbosity level           & Description \\
      \midrule
      \verb!none!               & No output \\
      \verb!low!                & Errors, important warnings, and some statistics \\
      \verb!medium!             & Same as \verb!low!, but with more statistics \\
      \verb!high!               & Errors, all warnings, and all statistics \\
      \verb!extreme!            & Same as \verb!high!, but also includes output from other packages (\textit{i.e.}, \zoltan{}) \\
      \bottomrule
    \end{tabular}
    \caption{Verbosity levels.}
\label{t:verbosity_types}
  \end{center}
\end{table}

\begin{table}[h!]
  \begin{center}
    \begin{tabular}{p{4.3cm} p{4.3cm} c p{4.5cm}}
      \toprule
      Problem type               & Multigrid algorithm    & Block size  & Smoother \\
      \midrule
      \verb!unknown!             & --                     & --          & -- \\
      \verb!Poisson-2D!          & Smoothed aggregation   & 1           & Chebyshev \\
      \verb!Poisson-3D!          & Smoothed aggregation   & 1           & Chebyshev \\
      \verb!Elasticity-2D!       & Smoothed aggregation   & 2           & Chebyshev \\
      \verb!Elasticity-3D!       & Smoothed aggregation   & 3           & Chebyshev \\
      \verb!ConvectionDiffusion! & Petrov-Galerkin  AMG   & 1           & Gauss-Seidel \\
      \verb!MHD!                 & Unsmoothed aggregation & --          & Additive Schwarz method with one level of overlap and ILU(0) as a subdomain solver \\
      \bottomrule
    \end{tabular}
    \caption{Supported problem types (``--'' means not set).}
\label{t:problem_types}
  \end{center}
\end{table}


\cbb{problem: type}{string}{"unknown"}{Type of problem to be solved. Possible values: see Table~\ref{t:problem_types}.}
          
\cbb{verbosity}{string}{"high"}{Control of the amount of printed information. Possible values: see Table~\ref{t:verbosity_types}.}
          
\cbb{number of equations}{int}{1}{Number of PDE equations at each grid node. Only constant block size is considered.}
          
\cbb{max levels}{int}{10}{Maximum number of levels in a hierarchy.}
          
\cbb{cycle type}{string}{"V"}{Multigrid cycle type. Possible values: "V", "W".}
          
\cbb{problem: symmetric}{bool}{true}{Symmetry of a problem. This setting affects the construction of a restrictor. If set to true, the restrictor is set to be the transpose of a prolongator. If set to false, underlying multigrid algorithm makes the decision.}
          
\cbb{xml parameter file}{string}{""}{An XML file from which to read additional
      parameters.  In case of a conflict, parameters manually set on
      the list will override parameters in the file. If the string is
      empty a file will not be read.}
          

% ==================== SMOOTHERS ====================
\section{Smoothing and coarse solver options}
\label{sec:options_smoothing}

\muelu{} relies on other \trilinos{} packages to provide level smoothers and
coarse solvers. \ifpack{} and \ifpacktwo{} provide standard smoothers (see
Table~\ref{tab:smoothers}), and \amesos{} and \amesostwo{} provide direct
solvers (see Table~\ref{tab:coarse}). Note that it is completely possible to use
any level smoother as a direct solver.

\muelu{} checks parameters \verb|smoother: * type| and \verb|coarse: type| to
determine:
\begin{itemize}
  \item what package to use (i.e., is it a smoother or a direct solver);
  \item (possibly) transform a smoother type

    \ding{42} \ifpack{} and \ifpacktwo{} use different smoother type names,
    e.g., ``point relaxation stand-alone'' vs ``RELAXATION''.  \muelu{} tries to follow
    \ifpacktwo{} notation for smoother types. Please consult \ifpacktwo{}
    documentation~\cite{Ifpack2URL} for more information.
\end{itemize}
The parameter lists \verb|smoother: * params| and \verb|coarse: params| are
passed directly to the corresponding package without any examination of their
content. Please consult the documentation of the corresponding packages for a list of
possible values.

By default, \muelu{} uses one sweep of symmetric Gauss-Seidel for both pre- and
post-smoothing, and SuperLU for coarse system solver.

\begin{table}[tbh]
  \begin{center}
    \begin{tabular}{p{4.0cm} p{10cm}}
      \toprule
      \texttt{smoother: type}           & \\
      \midrule
      \verb|RELAXATION|                 & Point relaxation smoothers, including
                                          Jacobi, Gauss-Seidel, symmetric Gauss-Seidel, etc. The exact
                                          smoother is chosen by specifying \texttt{relaxation: type} parameter in
                                          the \texttt{smoother: params} sublist. \\
      \verb|CHEBYSHEV|                  & Chebyshev polynomial smoother. \\
      \verb|ILUT|, \verb|RILUK|         & Local (processor-based) incomplete factorization methods. \\
      \bottomrule
    \end{tabular}
    \caption{Commonly used smoothers provided by \ifpack{}/\ifpacktwo{}. Note
    that these smoothers can also be used as coarse grid solvers.}
\label{tab:smoothers}
  \end{center}
\end{table}

\begin{table}[tbh]
  \begin{center}
    \begin{tabular}{p{4.0cm} c c p{7cm}}
      \toprule
      \texttt{coarse: type}             & \amesos{} & \amesostwo{} &  \\
      \midrule
      \verb|KLU|                        & x & & Default \amesos{} solver~\cite{klu}. \\
      \verb|KLU2|                       & & x & Default \amesostwo{} solver~\cite{amesos2_belos}. \\
      \verb|SuperLU|                    & x & x & Third-party serial sparse direct solver~\cite{Li2011}. \\
      \verb|SuperLU_dist|               & x & x & Third-party parallel sparse direct solver~\cite{Li2011}. \\
      \verb|Umfpack|                    & x & & Third-party solver~\cite{umfpack}. \\
      \verb|Mumps|                      & x & & Third-party solver~\cite{mumps}. \\
      \bottomrule
    \end{tabular}
    \caption{Commonly used direct solvers provided by \amesos{}/\amesostwo{}}
\label{tab:coarse}
  \end{center}
\end{table}


\cbb{smoother: pre or post}{string}{"both"}{Pre- and post-smoother combination. Possible values: "pre" (only pre-smoother), "post" (only post-smoother), "both" (both pre-and post-smoothers), "none" (no smoothing).}
          
\cbb{smoother: type}{string}{"RELAXATION"}{Smoother type. Possible values: see Table~\ref{tab:smoothers}.}
          
\cbb{smoother: pre type}{string}{"RELAXATION"}{Pre-smoother type. Possible values: see Table~\ref{tab:smoothers}.}
          
\cbb{smoother: post type}{string}{"RELAXATION"}{Post-smoother type. Possible values: see Table~\ref{tab:smoothers}.}
          
\cba{smoother: params}{\parameterlist}{Smoother parameters. For standard smoothers, \muelu passes them directly to the appropriate package library.}
          
\cba{smoother: pre params}{\parameterlist}{Pre-smoother parameters. For standard smoothers, \muelu passes them directly to the appropriate package library.}
          
\cba{smoother: post params}{\parameterlist}{Post-smoother parameters. For standard smoothers, \muelu passes them directly to the appropriate package library.}
          
\cbb{smoother: overlap}{int}{0}{Smoother subdomain overlap.}
          
\cbb{smoother: pre overlap}{int}{0}{Pre-smoother subdomain overlap.}
          
\cbb{smoother: post overlap}{int}{0}{Post-smoother subdomain overlap.}
          
\cbb{coarse: max size}{int}{2000}{Maximum dimension of a coarse grid. \muelu will stop coarsening once it is achieved.}
          
\cbb{coarse: type}{string}{"SuperLU"}{Coarse solver. Possible values: see Table~\ref{tab:coarse_solvers}.}
          
\cba{coarse: params}{\parameterlist}{Coarse solver parameters. \muelu passes them directly to the appropriate package library.}
          
\cbb{coarse: overlap}{int}{0}{Coarse solver subdomain overlap.}
          

% ==================== AGGREGATION ====================
\section{Aggregation options}
\label{sec:options_aggregation}

\begin{table}[h!]
  \begin{center}
    \begin{tabular}{p{5.0cm} p{10cm}}
      \toprule
      \verb!uncoupled! & Attempts to construct aggregates of optimal size ($3^d$
                         nodes in $d$ dimensions). Each process works independently, and
                         aggregates cannot span multiple processes.\\
      \verb!coupled!   & Attempts to construct aggregates of optimal size ($3^d$
                         nodes in $d$ dimensions). Aggregates are allowed to
                         cross processor boundaries. \textbf{Use carefully}. If
                         unsure, use \verb!uncoupled! instead.\\
      \verb!brick!     & Attempts to construct rectangular aggregates \\
      %\verb!METIS!     & Use graph partitioning algorithm to create aggregates,
      %                   working process-wise. Number of nodes in each aggregate
      %                   is specified with option \texttt{aggregation: max agg
      %                   size}. \\
      % \verb!ParMETIS!  & As \verb!METIS!, but partition global graph. Aggregates
                         % can span arbitrary number of processes. Specify global
                         % number of aggregates with {\tt aggregation: global
                         % number}. \\
      \bottomrule
    \end{tabular}
    \caption{Available coarsening schemes. }
\label{t:aggregation}
  \end{center}
\end{table}


\cbb{aggregation: type}{string}{"uncoupled"}{Aggregation scheme. Possible values: see Table~\ref{t:aggregation}.}
          
\cbb{aggregation: ordering}{string}{"natural"}{Node ordering strategy. Possible values: "natural" (local index order), "graph" (filtered graph breadth-first order), "random" (random local index order).}
          
\cbb{aggregation: drop scheme}{string}{"classical"}{Connectivity dropping scheme for a graph used in aggregation. Possible values: "classical", "distance laplacian".}
          
\cbb{aggregation: drop tol}{double}{0.0}{Connectivity dropping threshold for a graph used in aggregation.}
          
\cbb{aggregation: min agg size}{int}{2}{Minimum size of an aggregate.}
          
\cbb{aggregation: max agg size}{int}{-1}{Maximum size of an aggregate (-1 means unlimited).}
          
\cbb{aggregation: brick x size}{int}{2}{Number of points for x axis in "brick" aggregation (limited to 3).}
          
\cbb{aggregation: brick y size}{int}{2}{Number of points for y axis in "brick" aggregation (limited to 3).}
          
\cbb{aggregation: brick z size}{int}{2}{Number of points for z axis in "brick" aggregation (limited to 3).}
          
\cbb{aggregation: Dirichlet threshold}{double}{0.0}{Threshold for determining whether entries are zero during Dirichlet row detection.}
          
\cbb{aggregation: export visualization data}{bool}{false}{Export data for visualization post-processing.}
          
\cbb{aggregation: output filename}{string}{AggViz.vtp}{Filename to write VTK visualization data to.}
          
\cbb{aggregation: output file: time step}{int}{0}{Time step ID for non-linear problems.}
          
\cbb{aggregation: iter}{int}{0}{Iteration for non-linear problems.}
          

% ==================== REBALANCING ====================
\section{Rebalancing options}
\label{sec:options_rebalancing}


\cbb{repartition: enable}{bool}{false}{Rebalancing on/off switch.}
          
\cbb{repartition: partitioner}{string}{"zoltan2"}{Partitioning package to use. Possible values: "zoltan" (\zoltan{} library), "zoltan2" (\zoltantwo{} library).}
          
\cba{repartition: params}{\parameterlist}{Partitioner parameters. \muelu passes them directly to the appropriate package library.}
          
\cbb{repartition: start level}{int}{2}{Minimum level to run partitioner. \muelu does not rebalance levels finer than this one.}
          
\cbb{repartition: min rows per proc}{int}{800}{Minimum number of rows per processor. If the actual number if smaller, then rebalancing occurs.}
          
\cbb{repartition: max imbalance}{double}{1.2}{Maximum nonzero imbalance ratio. If the actual number is larger, the rebalancing occurs.}
          
\cbb{repartition: remap parts}{bool}{true}{Postprocessing for partitioning to reduce data migration.}
          
\cbb{repartition: rebalance P and R}{bool}{false}{Explicit rebalancing of R and P during the setup. This speeds up the solve, but slows down the setup phases.}
          

% ==================== MULTIGRID ====================
\section{Multigrid algorithm options}
\label{sec:options_mg}

\begin{table}[h!]
  \begin{center}
    \begin{tabular}{p{3.5cm} p{11cm}}
      \toprule
      \verb!sa!         & Classic smoothed aggregation~\cite{VMB1996} \\
      \verb!unsmoothed! & Aggregation-based, same as \verb!sa! but without damped Jacobi prolongator improvement step \\
      \verb!pg!         & Prolongator smoothing using $A$, restriction smoothing using $A^T$, local damping factors~\cite{ST2008} \\
      \verb!emin!       & Constrained minimization of energy in basis functions of grid transfer operator~\cite{WTWG2014,OST2011} \\
      \bottomrule
    \end{tabular}
    \caption{Available multigrid algorithms for generating grid transfer matrices. }
\label{t:mgs}
  \end{center}
\end{table}


\cbb{multigrid algorithm}{string}{"sa"}{Multigrid method. Possible values: see Table~\ref{t:mgs}.}
          
\cbb{semicoarsen: coarsen rate}{int}{3}{Rate at which to coarsen unknowns in the z direction.}
          
\cbb{sa: damping factor}{double}{1.33}{Damping factor for smoothed aggregation.}
          
\cbb{sa: use filtered matrix}{bool}{true}{Matrix to use for smoothing the tentative prolongator. The two options are: to use the original matrix, and to use the filtered matrix with filtering based on filtered graph used for aggregation.}
          
\cbb{filtered matrix: use lumping}{bool}{true}{Lump (add to diagonal) dropped entries during the construction of a filtered matrix. This allows user to preserve constant nullspace.}
          
\cbb{filtered matrix: reuse eigenvalue}{bool}{true}{Skip eigenvalue calculation during the construction of a filtered matrix by reusing eigenvalue estimate from the original matrix. This allows us to skip heavy computation, but may lead to poorer convergence.}
          
\cbb{emin: iterative method}{string}{"cg"}{Iterative method to use for energy minimization of initial prolongator in energy-minimization. Possible values: "cg" (conjugate gradient), "sd" (steepest descent).}
          
\cbb{emin: num iterations}{int}{2}{Number of iterations to minimize initial prolongator energy in energy-minimization.}
          
\cbb{emin: num reuse iterations}{int}{1}{Number of iterations to minimize the reused prolongator energy in energy-minimization.}
          
\cbb{emin: pattern}{string}{"AkPtent"}{Sparsity pattern to use for energy minimization. Possible values: "AkPtent".}
          
\cbb{emin: pattern order}{int}{1}{Matrix order for the "AkPtent" pattern.}
          

% ==================== REUSE ====================
\section{Reuse options}
\label{sec:options_reuse}

Reuse options are currently only used with \verb!sa! multigrid algorithm. We
also assume that the matrix preserves graph structure, and only matrix values
change.

In addition, please note that not all combinations of multigrid algorithms and
reuse options are valid, or even make sense. For instance, the "emin" reuse
option should only be use with "emin" multigrid algorithm.

\begin{table}[h!]
  \begin{center}
    \begin{tabular}{p{3.0cm} p{12cm}}
      \toprule
      \verb!none!   & No reuse \\
      \verb!emin!   & Reuse old prolongator as an initial guess to energy
                      minimization, and reuse the prolongator pattern \\
      \verb!RP!     & Reuse smoothed prolongator and restrictor. Smoothers are
                      recomputed.  \ding{42} \verb!RP! should reuse matrix graphs for
                      matrix-matrix product, but currently that is disabled as only \epetra{}
                      supports it. \\
      \verb!tP!     & Reuse tentative prolongator. The graphs of smoothed
                      prolongator and matrices in Galerkin product are reused
                      only if filtering is not being used ({\it i.e.}, either
                      \verb!sa: use filtered matrix! or \verb!aggregation: drop tol! is false) \\
      \verb!RAP!    & Recompute only the finest level smoothers, reuse all other operators \\
      \verb!full!   & Reuse everything \\
      \bottomrule
    \end{tabular}
    \caption{Available coarsening schemes. }
\label{t:reuse_types}
  \end{center}
\end{table}


\cbb{reuse: type}{string}{"none"}{Reuse options for consecutive hierarchy construction. This speeds up the setup phase, but may lead to poorer convergence. Possible values: see Table~\ref{t:reuse_types}.}
          

% ==================== MISCELLANEOUS ====================
\section{Miscellaneous options}


\cba{export data}{\parameterlist}{Exporting a subset of the hierarchy data in a file. Currently, the list can contain any of three parameter names ("A", ``P'', ``R'') of type \texttt{string} and value ``\{levels separated by commas\}''. A matrix with a name ``X'' is saved in three MatrixMarket files: a) data is saved in \textit{X\_level.mm}; b) its row map is saved in \textit{rowmap\_X\_level.mm}; c) its column map is saved in \textit{colmap\_X\_level.mm}.}
          
\cbb{print initial parameters}{bool}{true}{Print parameters provided for a hierarchy construction.}
          
\cbb{print unused parameters}{bool}{true}{Print parameters unused during a hierarchy construction.}
          
\cbb{transpose: use implicit}{bool}{false}{Use implicit transpose for the restriction operator.}
          


    %-----------------------------%
    \chapter{\muemex: The MATLAB Interface for \muelu} \label{sec:muemex}
    %-----------------------------%
    %%%%%%%%%%%%%%%%%%%%%%%%%%%%%%%%%%%%%%%%%%%%%%%%%%%%%%%%%%%%%%%%%%%
\muemex is \muelu's interface to the MATLAB environment. It allows access
to a limited set of routines either \muelu as a preconditioner,
Belos as a solver and Epetra or Tpetra for data structures.
It is designed to provide access to \muelu's aggregation and
solver routines from MATLAB and does little else. \muemex allows users to
setup and solve arbitrarily many problems, so long as memory suffices.
More than one problem can be set up simultaneously.

\section{Cmake Configure and Make}\label{sec:muemex:cmake}
To use \muemex, Trilinos must be configured with (at least) the
following options:

\begin{lstlisting}
  export TRILINOS_HOME=/path/to/your/Trilinos/source/directory
  cmake \
      -D Trilinos_ENABLE_Amesos:BOOL=ON \
      -D Trilinos_ENABLE_Amesos2:BOOL=ON \
      -D Amesos2_ENABLE_KLU2:BOOL=ON \
      -D Trilinos_ENABLE_AztecOO:BOOL=ON \
      -D Trilinos_ENABLE_Epetra:BOOL=ON \
      -D Trilinos_ENABLE_EpetraExt:BOOL=ON \
      -D Trilinos_ENABLE_Fortran:BOOL=OFF \
      -D Trilinos_ENABLE_Ifpack:BOOL=ON \
      -D Trilinos_ENABLE_Ifpack:BOOL=ON \
      -D Trilinos_ENABLE_MueLu:BOOL=ON \
      -D Trilinos_ENABLE_Teuchos:BOOL=ON \
      -D Trilinos_ENABLE_Tpetra:BOOL=ON \
      -D TPL_ENABLE_MPI:BOOL=OFF \
      -D TPL_ENABLE_MATLAB:BOOL=ON \
      -D MATLAB_ROOT:STRING="<my matlab root>" \
      -D MATLAB_ARCH:STRING="<my matlab os string>" \
  ${TRILINOS_HOME}
\end{lstlisting}

Since \muemex supports both the Epetra and Tpetra linear algebra
libraries, you have to have both enabled in order to build \muemex.
Most additional options can be specified as well.  It is important to
note that \muemex does not work properly with MPI, hence MPI must be
disabled in order to compile \muemex.  The MATLAB\_ARCH option is new to
the cmake build system, and involves the MATLAB-specific architecture
code for your system.  There is currently no automatic way to extract
this, so it must be user-specified.  As of MATLAB 7.9 (R2009b), common
arch codes are:
\begin{center}
\begin{tabular}{l|l}
Code& OS\\
\hline
glnx86& 32-bit Linux (intel/amd)\\
glnxa64& 64-bit Linux (intel/amd)\\
maci64& 64-bit MacOS\\
maci& 32-bit MacOS\\
\end{tabular}
\end{center}

On 64-bit Intel/AMD architectures, Trilinos and all relevant TPLs
(note: this includes BLAS and LAPACK)
must be compiled with the \texttt{-fPIC} option.  This necessitates adding:
\begin{lstlisting}
    -D CMAKE_CXX_FLAGS:STRING="-fPIC" \
    -D CMAKE_C_FLAGS:STRING="-fPIC" \
    -D CMAKE_Fortran_FLAGS:STRING="-fPIC" \
\end{lstlisting}
to the cmake configure line.

\subsection{BLAS \& LAPACK Option \#1: Static Builds}
Trilinos does not play nicely with MATLAB's default LAPACK and BLAS on
64-bit machines.
If \muemex randomly crashes when you run with any Krylov method that
has orthogonalization, chances are \muemex is finding the wrong
BLAS/LAPACK libraries.
This leaves you
with one of two options.  The first is to build them both \textit{statically}
and then specify them as follows:
\begin{lstlisting}
    -D LAPACK_LIBRARY_DIRS:STRING="<path to my lapack.a>" \
    -D BLAS_LIBRARY_DIRS:STRING="<path to my blas.a>" \
\end{lstlisting}
Using static linking for LAPACK and BLAS prevents MATLAB's default libraries to take precedence.

\subsection{BLAS \& LAPACK Option \#2: LD$\_$PRELOAD}
The second option is to use \textsf{LD\_PRELOAD} to tell MATLAB exactly
which libraries to use.  For this option, you can use the dynamic
libraries installed on your system.
Before starting MATLAB, set
LD\_PRELOAD to the paths of libstdc++.so corresponding to the version of GCC used
to build Trilinos, and the paths of libblas.so and liblapack.so on your local system.

For example, if you use bash, you'd do something like this
\begin{lstlisting}
  export LD_PRELOAD=<path>/libstdc++.so:<path>/libblas.so:<path>/liblapack.so
  \end{lstlisting}

For csh / tcsh, do this
\begin{lstlisting}
  setenv LD_PRELOAD <path>/libstdc++.so:<path>/libblas.so:<path>/liblapack.so
\end{lstlisting}

\section{Using \muemex}\label{sec:muemex:usage}
\muemex is designed to be interfaced with via the MATLAB script
\texttt{muelu.m}.  There are five modes in which \muemex can be run:
\begin{enumerate}
\item Setup Mode --- Performs the problem setup for \muelu.
  Depending on whether or not the \texttt{Linear Algebra} option is
  used, \muemex creates either an unpreconditioned Epetra problem,
  an Epetra problem with \muelu, or a Tpetra problem with \muelu.
  The default is \texttt{tpetra}. The \texttt{epetra} mode only supports
  real-valued matrices, while \texttt{tpetra} 
  supports both real and complex and will infer the scalar type
  from the matrix passed during setup.  This call returns a problem
  handle used to reference the problem in the future, and (optionally)
  the operator complexity, if a preconditioner is being used.
\item Solve Mode --- Given a problem handle and a right-hand side, \muemex
  solves the problem specified.  Setup mode must be called before
  solve mode.
\item Cleanup Mode --- Frees the memory allocated to internal \muelu,
  Epetra and Tpetra objects.  This can be called with a particular
  problem handle, in which case it frees that problem, or without one,
  in which case all \muemex memory is freed.
\item Status Mode --- Prints out status information on problems which
  have been set up.  Like cleanup, it can be called with or without a
  particular problem handle.
\item Get Mode --- Get information from a MueLu hierarchy that has been
  generated. Given the problem handle, a level number and the name of the
  field, returns the appropriate array or scalar as a MATLAB object.
\end{enumerate}
All of these modes, with the exception of status and cleanup take
option lists which will be directly converted into
\texttt{Teuchos::ParameterList} objects by \muemex, as key-value pairs.
Options passed during setup will apply to the \muelu preconditioner, and
options passed during a solve will apply to Belos.

\subsection{Setup Mode}
Setup mode is called as follows:
\begin{lstlisting}[language=Matlab]
  >> [h, oc] = muelu('setup', A[, 'parameter', value,...])
\end{lstlisting}
The parameter \texttt{A} represents the sparse matrix to perform aggregation on
and the parameter/value pairs represent standard \muelu options.

The routine returns a problem handle, \texttt{h}, and the operator
complexity \texttt{oc} for the operator.  In addition to the standard
options, setup mode has one unique option of its own:

\choicebox{\tt Linear Algebra}{[{\tt string}] Whether to use
  'epetra unprec', 'epetra', or 'tpetra'. Default is 'epetra' for
  real matrix and 'tpetra' for complex matrix.}

\subsection{Solve Mode}
Solve mode is called as follows:
\begin{lstlisting}[language=Matlab]
  >> [x, its] = muelu(h[, A], b[, 'parameter', value,...])
\end{lstlisting}
The parameter \texttt{h} is a problem handle returned by the
setup mode call, \texttt{A} is the sparse matrix with which to
solve and \texttt{b} is the right-hand side.  Parameter/value pairs
to configure the Belos solver are listed as above. If A is not supplied,
the matrix provided when setting up the problem will be used. \texttt{x} is
the solution multivector with the same dimensions as \texttt{b}, and \texttt{its}
is the number of iterations Belos needed to solve the problem.

All of these options are taken directly from Belos, so consult its
manual for more information. Belos output style and verbosity settings
are implemented as enums, but can be set as strings in \muemex. For example:

\begin{lstlisting}[language=Matlab]
  >> x = muelu(0, b, 'Verbosity', 'Warnings + IterationDetails', ...
                       'Output Style', 'Brief');
\end{lstlisting}

Verbosity settings can be separated by spaces, '+' or ','. Belos::Brief
is the default output style.

\subsection{Cleanup Mode}
Cleanup mode is called as follows:
\begin{lstlisting}[language=Matlab]
  >> muelu('cleanup'[, h])
\end{lstlisting}
The parameter \texttt{h} is a problem handle returned by the
setup mode call and is optional.  If \texttt{h} is provided, that
problem is cleaned up.  If the option is not provided all currently
set up problems are cleaned up.

\subsection{Status Mode}
Status mode is called as follows:
\begin{lstlisting}[language=Matlab]
  >> muelu('status'[, h])
\end{lstlisting}
The parameter \texttt{h} is a problem handle returned by the
setup mode call and is optional.  If \texttt{h} is provided, status
information for that problem is printed.  If the option is not provided all currently
set up problems have status information printed.

\subsection{Get Mode}
Get mode is called as follows:
\begin{lstlisting}[language=Matlab]]
  >> muelu('get', h, level, fieldName[, typeHint])
\end{lstlisting}
The parameter \texttt{h} is the problem handle, and \texttt{level}
is an integer that identifies the level within the hierarchy containing
the desired data. \texttt{fieldName} is a string that identifies the
field within the level, e.g. 'Nullspace'. \texttt{typeHint} is an optional
parameter that tells MueMex what data type to expect from the level. This
is a string, with possible values 'matrix', 'multivector', 'lovector' (ordinal
vector), or 'scalar'. MueMex will attempt to guess the type from \texttt{fieldName}
but \texttt{typeHint} may be required.

\subsection{Tips and Tricks }\label{sec:muemex:tips}

Internally, MATLAB represents all data as doubles unless you go
through efforts to do otherwise.  \muemex detects integer parameters by
a relative error test, seeing if the relative difference between the
value from MATLAB and the value of the \texttt{int}-typecast value are
less than 1e-15.  Unfortunately, this means that \muemex will choose the
incorrect type for parameters which are doubles that happen to have an
integer value (a good example of where this might happen would be the parameter
`smoother Chebyshev: alpha', which defaults to 30.0).  Since \muemex does no
internal typechecking of
parameters (it uses \muelu's internal checks), it has no way of detecting
this conflict.  From the user's perspective, avoiding this is as
simple as adding a small perturbation (greater than a relative 1e-15)
to the parameter that makes it non-integer valued.


    %\nocite{*}

    % ---------------------------------------------------------------------- %
    % References
    %
    \clearpage
    % If hyperref is included, then \phantomsection is already defined.
    % If not, we need to define it.
    \providecommand*{\phantomsection}{}
    \phantomsection
    \addcontentsline{toc}{chapter}{References}
    \bibliographystyle{plain}
    \bibliography{mueluguide}


    % ---------------------------------------------------------------------- %
    %
    \appendix
    \chapter{Copyright and License}
    \label{sec:license}
\begin{center}
MueLu: A package for multigrid based preconditioning

Copyright 2012 Sandia Corporation
\end{center}

\noindent
Under the terms of Contract DE--AC04--94AL85000 with Sandia Corporation,
the U.S. Government retains certain rights in this software.

\noindent
Redistribution and use in source and binary forms, with or without
modification, are permitted provided that the following conditions are
met:

\begin{enumerate}
  \item Redistributions of source code must retain the above copyright
    notice, this list of conditions and the following disclaimer.

\item Redistributions in binary form must reproduce the above copyright
  notice, this list of conditions and the following disclaimer in the
  documentation and/or other materials provided with the distribution.

\item Neither the name of the Corporation nor the names of the
  contributors may be used to endorse or promote products derived from
  this software without specific prior written permission.
\end{enumerate}

\noindent
THIS SOFTWARE IS PROVIDED BY SANDIA CORPORATION ``AS IS'' AND ANY
EXPRESS OR IMPLIED WARRANTIES, INCLUDING, BUT NOT LIMITED TO, THE
IMPLIED WARRANTIES OF MERCHANTABILITY AND FITNESS FOR A PARTICULAR
PURPOSE ARE DISCLAIMED\@. IN NO EVENT SHALL SANDIA CORPORATION OR THE
CONTRIBUTORS BE LIABLE FOR ANY DIRECT, INDIRECT, INCIDENTAL, SPECIAL,
EXEMPLARY, OR CONSEQUENTIAL DAMAGES (INCLUDING, BUT NOT LIMITED TO,
PROCUREMENT OF SUBSTITUTE GOODS OR SERVICES\@; LOSS OF USE, DATA, OR
PROFITS\@; OR BUSINESS INTERRUPTION) HOWEVER CAUSED AND ON ANY THEORY OF
LIABILITY, \\WHETHER IN CONTRACT, STRICT LIABILITY, OR TORT (INCLUDING
NEGLIGENCE OR OTHERWISE) ARISING IN ANY WAY OUT OF THE USE OF THIS
SOFTWARE, EVEN IF ADVISED OF THE POSSIBILITY OF SUCH DAMAGE\@.

    %\chapter{Historical Perspective}
	%\input{CommonHistory}

    \chapter{ML compatibility}
     
\label{sec:ml_options}
\muelu provides a basic compatibility layer for \ml parameter lists. This allows \ml users to quickly perform some experiments with \muelu. 

\textbf{First and most important: } Long term, we would like to have users use the new \muelu interface, as that is where most of new features will be made accessible. One should make note of the fact that it may not be possible to make ML deck do exactly same things in \ml and \muelu, as internally \ml implicitly makes some decision that we have no control over and which could be different from \muelu.

\noindent There are basically two distinct ways to use \ml input parameters with \muelu:
\begin{description}
\item[MLParameterListInterpreter:] This class is the pendant of the \texttt{ParameterListInterpreter} class for the \muelu parameters. It accepts parameter lists or XML files with \ml parameters and generates a \muelu multigrid hierarchy. It supports only a well-defined subset of \ml parameters which have a equivalent parameter in \muelu.
\item[ML2MueLuParameterTranslator:] This class is a simple wrapper class which translates \ml parameters to the corresponding \muelu parameters. It has to be used in combination with the \muelu \texttt{ParameterListInterpreter} class to generate a \muelu multigrid hierarchy. It is also meant to be used in combination with the \texttt{CreateEpetraPreconditioner} and \texttt{CreateTpetraPreconditioner} routines (see \S\ref{sec:examples in code}). It supports only a small subset of the \ml parameters.
\end{description}

\section{Usage of \ml parameter lists with \muelu}

\subsection{MLParameterListInterpreter}

The \texttt{MLParameterListInterpreter} directly accepts a \texttt{ParameterList} containing \ml parameters. It also interprets the \texttt{null space: vectors} and the \texttt{null space: dimension} \ml parameters. However, it is recommended to provide the near null space vectors directly in the \muelu way as shown in the following code snippet.

\begin{lstlisting}[language=C++]
    Teuchos::RCP<Tpetra::CrsMatrix<> > A;
    // create A here ...
    
    // XML file containing ML parameters
    std::string xmlFile = "mlParameters.xml"
    Teuchos::ParameterList paramList;
    Teuchos::updateParametersFromXmlFileAndBroadcast(xmlFile, Teuchos::Ptr<Teuchos::ParameterList>(&paramList), *comm);
       
    // use ParameterListInterpreter with MueLu parameters as input
    Teuchos::RCP<HierarchyManager> mueluFactory = Teuchos::rcp(new MLParameterListInterpreter(*paramList));
    
    RCP<Hierarchy> H = mueluFactory->CreateHierarchy();
    H->GetLevel(0)->Set<RCP<Matrix> >("A", A);
    H->GetLevel(0)->Set("Nullspace", nullspace);
    H->GetLevel(0)->Set("Coordinates", coordinates);
    mueluFactory->SetupHierarchy(*H);
\end{lstlisting}

Note that the \texttt{MLParameterListInterpreter} only supports a basic set of \ml parameters allowing to build smoothed aggregation transfer operaotrs (see \S\ref{sec:compatiblemlparameters} for a list of compatible \ml parameters).

\subsection{ML2MueLuParameterTranslator}

The \texttt{Ml2MueLuParameterTranslator} class is a simple wrapper translating \ml parameters to the corresponding \muelu parameters. This allows the usage of the simple \texttt{CreateEpetraPreconditioner} and \texttt{CreateTpetraPreconditioner} interface with \ml parameters:

\begin{lstlisting}[language=C++]
    Teuchos::RCP<Tpetra::CrsMatrix<> > A;
    // create A here ...
    
    // XML file containing ML parameters
    std::string xmlFile = "mlParameters.xml"
    Teuchos::ParameterList paramList;
    Teuchos::updateParametersFromXmlFileAndBroadcast(xmlFile, Teuchos::Ptr<Teuchos::ParameterList>(&paramList), *comm);

    // translate ML parameters to MueLu parameters
    RCP<ParameterList> mueluParamList = Teuchos::getParametersFromXmlString(MueLu::ML2MueLuParameterTranslator::translate(paramList,"SA"));

    Teuchos::RCP<MueLu::TpetraOperator> mueLuPreconditioner =
       MueLu::CreateTpetraPreconditioner(A, mueluParamList);
\end{lstlisting}

In a similar way, \ml input parameters can be used with the standard \muelu parameter list interpreter class. Note that the near null space vectors have to be provided in the \muelu way.

\begin{lstlisting}[language=C++]
    Teuchos::RCP<Tpetra::CrsMatrix<> > A;
    // create A here ...
    
    // XML file containing ML parameters
    std::string xmlFile = "mlParameters.xml"
    Teuchos::ParameterList paramList;
    Teuchos::updateParametersFromXmlFileAndBroadcast(xmlFile, Teuchos::Ptr<Teuchos::ParameterList>(&paramList), *comm);
    
    // translate ML parameters to MueLu parameters
    RCP<ParameterList> mueluParamList = Teuchos::getParametersFromXmlString(MueLu::ML2MueLuParameterTranslator::translate(paramList,"SA"));
    
    // use ParameterListInterpreter with MueLu parameters as input
    Teuchos::RCP<HierarchyManager> mueluFactory = Teuchos::rcp(new ParameterListInterpreter(*mueluParamList));
    
    RCP<Hierarchy> H = mueluFactory->CreateHierarchy();
    H->GetLevel(0)->Set<RCP<Matrix> >("A", A);
    H->GetLevel(0)->Set("Nullspace", nullspace);
    H->GetLevel(0)->Set("Coordinates", coordinates);
    mueluFactory->SetupHierarchy(*H);
\end{lstlisting}

Note that the set of supported \ml parameters is very limited. Please refer to \S\ref{sec:compatiblemlparameters} for a list of all compatible \ml parameters.

\section{Compatible \ml parameters}
\label{sec:compatiblemlparameters}
\subsection{General \ml options}

\mlcbb{ML output}{int}{0}{MLParameterListInterpreter, ML2MueLuParameterTranslator}{Control of the amount of printed information. Possible values: 0-10 with 0=no output and 10=maximum verbosity.}   
      
\mlcbb{PDE equations}{int}{1}{MLParameterListInterpreter, ML2MueLuParameterTranslator}{Number of PDE equations at each grid node. Only constant block size is considered.}   
       
\mlcbb{max levels}{int}{10}{MLParameterListInterpreter, ML2MueLuParameterTranslator}{Maximum number of levels in a hierarchy.}   
      
\mlcbb{prec type}{string}{"MGV"}{MLParameterListInterpreter, ML2MueLuParameterTranslator}{Multigrid cycle type. Possible values: "MGV", "MGW". Other values are NOT supported by MueLu.}   
      

\subsection{Smoothing and coarse solver options}

\mlcbb{smoother: type}{string}{"symmetric Gauss-Seidel"}{MLParameterListInterpreter, ML2MueLuParameterTranslator}{Smoother type for fine- and intermedium multigrid levels. Possible values: "Jacobi", "Gauss-Seidel", "symmetric Gauss-Seidel", "Chebyshev", "ILU".}

\mlcbb{smoother: sweeps}{int}{2}{MLParameterListInterpreter, ML2MueLuParameterTranslator}{Number of smoother sweeps for relaxation based level smoothers. In case of Chebyshev smoother it denotes the polynomial degree.}

\mlcbb{smoother: damping factor}{double}{1.0}{MLParameterListInterpreter, ML2MueLuParameterTranslator}{Damping factor for relaxation based level smoothers.}

\mlcbb{smoother: Chebyshev alpha}{double}{20}{MLParameterListInterpreter, ML2MueLuParameterTranslator}{Eigenvalue ratio for Chebyshev level smoother.}

 
\mlcbb{smoother: pre or post}{string}{"both"}{MLParameterListInterpreter, ML2MueLuParameterTranslator}{Pre- and post-smoother combination. Possible values: "pre" (only pre-smoother), "post" (only post-smoother), "both" (both pre-and post-smoothers).}   
      
\mlcbb{max size}{int}{128}{MLParameterListInterpreter, ML2MueLuParameterTranslator}{Maximum dimension of a coarse grid. \ml will stop coarsening once it is achieved.}   
      

\mlcbb{coarse: type}{string}{"Amesos-KLU"}{MLParameterListInterpreter, ML2MueLuParameterTranslator}{Solver for coarsest level. Possible values: "Amesos-KLU", "Amesos-Superlu" (depending on \muelu installation).}


\subsection{Transfer operator options}

\mlcbb{energy minimization: enable}{int}{0}{MLParameterListInterpreter, ML2MueLuParameterTranslator}{Enable energy minimization transfer operators (using Petrov-Galerkin approach).}   
      
\mlcbb{aggregation: damping factor}{double}{1.33}{MLParameterListInterpreter, ML2MueLuParameterTranslator}{Damping factor for smoothed aggregation.}   
      

\subsection{Rebalancing options}

\mlcbb{repartition: enable}{int}{0}{MLParameterListInterpreter}{Rebalancing on/off switch. Only limited support for repartitioning. Does not use provided node coordinates.}   
      
\mlcbb{repartition: start level}{int}{1}{MLParameterListInterpreter}{Minimum level to run partitioner. \muelu does not rebalance levels finer than this one.}   
      
\mlcbb{repartition: min per proc}{int}{512}{MLParameterListInterpreter}{Minimum number of rows per processor. If the actual number if smaller, then rebalancing occurs.}   
      
\mlcbb{repartition: max min ratio}{double}{1.3}{MLParameterListInterpreter}{Maximum nonzero imbalance ratio. If the actual number is larger, the rebalancing occurs.}   
      



    %\chapter{Some Other Appendix}
	%\input{CommonAppendix}

    % \printindex

    %
% This is an example of how to create the distribution page. Some
% distributions are required by Sandia; e.g. the housekeeping copies.
% Depending on the type of report; e.g. CRADA, Patent Caution, etc.
% additional distribution lines may have to be added. See the
% "Guide for Preparing SAND Reports"
%
% SANDdistribution takes CA or NM as an optional argument. If given,
% the approrpiate housekeeping copies are inserted autmatically.
% Inside the SANDdistribution environment, several commands can be used
% insert the distributions for CRADA, LDRD, etc. See example below.
%
% You can leave the CA or NM option off and not use any of the SANDdist*
% commands. This will allow you to create a distribution list manually.
%
\begin{SANDdistribution}[NM]
    % Housekeeping copies necessary for every unclassified report:
    % \SANDdistCRADA	% If this report is about CRADA work
    % \SANDdistPatent	% If this report has a Patent Caution or Patent Interest
    % \SANDdistLDRD	% If this report is about LDRD work

    % Some external Addresses
    %\SANDdistExternal{1}{An Address\\ 99 $99^{th}$ street NW\\City, State}
    %\SANDdistExternal{3}{Some Address\\ and street\\City, State}
    %\SANDdistExternal{12}{Another Address\\ On a street\\City, State\\U.S.A.}
    \SANDdistExternal{1}{Tobias Wiesner\\Institute for Computational Mechanics \\Technische Universit\"at
    M\"unchen\\Boltzmanstra\ss e 15 \\85747 Garching, Germany}
    \bigskip


    % The following MUST BE between the external and internal distributions!
    % \SANDdistClassified % If this report is classified


    % Internal Addresses
    \SANDdistInternal{1}{1320}{Michael Heroux}{1446}
    \SANDdistInternal{1}{1318}{Robert Hoekstra}{1446}
    \SANDdistInternal{1}{1320}{Mark Hoemmen}{1446}
    \SANDdistInternal{1}{1320}{Paul Lin}{1446}
    \SANDdistInternal{1}{1318}{Andrey Prokopenko}{1426}
    \SANDdistInternal{1}{1322}{Christopher Siefert}{1443}

\end{SANDdistribution}

\begin{SANDdistribution}[CA]
    \SANDdistInternal{1}{9159}{Jonathan Hu}{1426}
    \SANDdistInternal{1}{9159}{Paul Tsuji}{1442}
    \SANDdistInternal{1}{9159}{Raymond Tuminaro}{1442}
\end{SANDdistribution}


\end{document}
